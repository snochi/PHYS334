\documentclass[phys334]{subfiles}

%% ========================================================
%% document

\begin{document}

    \section{Introduction}

    \subsection{Review of Postulates and Formalism}

    \begin{postulate}{}
        The state of a quantum system can be described by a wavefunction in a Hilbert space.
    \end{postulate}

    \np
    We denote the elements of the Hilbert space as $\left| \psi \right\rangle$, and the complex conjugate by $\left\langle \psi \right|$.

    \begin{postulate}{}
        Every observable/measurable quantity is described by an operator $\hat{A}$ on a Hilbert space.
    \end{postulate}

    \begin{postulate}{}
        The result of measuring an observable is one of its eigenvalues.
    \end{postulate}

    \begin{postulate}{Born's Rule}
        For an observable $\hat{A}$ and its eigenvalue $a$ correspoinding to $\left| a \right\rangle$, the probability of measuring $a$ is
        \begin{equation*}
            \PP\left( a \right) = \left| \left\langle a | \psi \right\rangle \right|^{2},
        \end{equation*}
        where $\left| \psi \right\rangle$ is the quantum state before the measurement.
    \end{postulate}

    \begin{postulate}{}
        For an observable $\hat{A}$, if an eigenvalue $a$ is measured, then the quantum state after the measurement is $\left| a \right\rangle$.
    \end{postulate}

    \begin{postulate}{Schrodinger Equation}
        The time evolution of a quantum state $\left| \psi \right\rangle$ satisfies
        \begin{equation*}
            i\hbar \frac{\partial}{\partial t} \left| \psi \right\rangle = H\left| \psi \right\rangle,
        \end{equation*}
        where $H$ is the Hamiltonian operator.
    \end{postulate}

    \subsection{Time Evolution Operator}

    Consider the time dependent Schrodinger equation
    \begin{equation*}
        i\hbar \frac{\partial}{\partial t} \left| \psi \right\rangle = H\left| \psi \right\rangle.
    \end{equation*}
    Then the state $\left| \psi \right\rangle$ evolves over time according to the \textit{time evolution operator} $U\left( t,t_0 \right)$ by
    \begin{equation*}
        \left| \psi\left( t \right) \right\rangle = U\left( t,t_0 \right) \left| \psi\left( t_0 \right) \right\rangle
    \end{equation*}
    with the initial condition $U\left( t_0,t_0 \right) = I$, the identity operator. Note that $U$ has to be unitary, since we want to preserve the norm. Therefore, the differential equation for $U$ is given by
    \begin{equation*}
        i\hbar \frac{\partial}{\partial t} U\left( t,t_0 \right)\left| \psi\left( t_0 \right) \right\rangle = HU\left( t,t_0 \right)\left| \psi\left( t_0 \right) \right\rangle.
    \end{equation*}
    Rewriting in terms of $U$,
    \begin{equation*}
        \frac{\partial}{\partial t} U\left( t,t_0 \right) = -\frac{i}{\hbar} HU\left( t,t_0 \right).
    \end{equation*}
    In case $H$ is independent of the time,
    \begin{equation*}
        U = Ae^{-\frac{i}{\hbar}Ht},
    \end{equation*}
    where $A$ is a constant depending on the initial condition $U\left( t_0,t_0 \right)=I$. That is,
    \begin{equation*}
        I = U\left( t_0,t_0 \right) = Ae^{-\frac{i}{\hbar}Ht_0} \implies A = e^{\frac{i}{\hbar}Ht_0}.
    \end{equation*}

    This solution is useful when paired with the \textit{time independent Schrodinger equation}:
    \begin{equation*}
        H\left| E_n \right\rangle = E_n\left| E_n \right\rangle.
    \end{equation*}
    In this case, for any analytic function $f$,
    \begin{equation*}
        f\left( H \right)\left| E_n \right\rangle = f\left( E_n \right)\left| E_n \right\rangle,
    \end{equation*}
    so in particular,
    \begin{equation*}
        U\left( t,t_0 \right)\left| E_n \right\rangle = e^{-\frac{i}{\hbar}E_n\left( t-t_0 \right)}\left| E_n \right\rangle.
    \end{equation*}
    For a general state $\left| \psi\left( t_0 \right) \right\rangle$, we write it down as a linear combination of energy eigenstates,
    \begin{equation*}
        \left| \psi\left( t_0 \right) \right\rangle = \left( \sum^{}_{n} \left| E_n \right\rangle \left\langle E_n \right| \right) \left| \psi\left( t_0 \right) \right\rangle = \sum^{}_{n} c_n\left| E_n \right\rangle,
    \end{equation*}
    where $c_n = \left\langle E_n | \psi\left( t_0 \right) \right\rangle$. Then,
    \begin{equation*}
\left| \psi\left( t \right) \right\rangle = U\left( t,t_0 \right)\left| \psi\left( t_0 \right) \right\rangle = e^{-\frac{i}{\hbar}H\left( t-t_0 \right)} \sum^{}_{n}c_n\left| E_n \right\rangle = \sum^{}_{n} c_n e^{-\frac{i}{\hbar}E_n\left( t-t_0 \right)} \left| E_n \right\rangle.
    \end{equation*}
    Thus
    \begin{equation*}
        U\left( t,t_0 \right) = \sum^{}_{n} e^{-\frac{i}{\hbar}E_n\left( t-t_0 \right)} \left| E_n \right\rangle \left\langle E_n \right|.
    \end{equation*}

    \begin{example}{Time Evolution of Spin-$\frac{1}{2}$}
        Let
        \begin{equation*}
            H = \hbar\Omega\sigma_x = \hbar\Omega \left( \left| \uparrow \right\rangle \left\langle \downarrow \right| + \left| \downarrow \right\rangle \left\langle \uparrow \right| \right) = \hbar\Omega
            \begin{bmatrix}
            	0 & 1 \\
            	1 & 0 \\
            \end{bmatrix},
        \end{equation*}
        where $\left| \uparrow \right\rangle = \begin{bmatrix} 1 \\ 0 \end{bmatrix} , \left| \downarrow \right\rangle = \begin{bmatrix} 0\\1 \end{bmatrix}$. Note that the energy eigenvectors are
        \begin{equation*}
            \begin{aligned}
                \left| E_1 \right\rangle & = \frac{1}{\sqrt{2}} \left( \left| \uparrow \right\rangle + \left| \downarrow \right\rangle \right) \text{ corresponding to $\hbar\Omega=E_1$} \\
                \left| E_2 \right\rangle & = \frac{1}{\sqrt{2}} \left( \left| \uparrow \right\rangle - \left| \downarrow \right\rangle \right) \text{ corresponding to $-\hbar\Omega=E_2$}
            \end{aligned} .
        \end{equation*}

        Consider
        \begin{equation*}
            \left| \psi\left( 0 \right) \right\rangle = \left| \downarrow \right\rangle.
        \end{equation*}
        Then the coefficients with respect to the basis $\left( \left| E_1 \right\rangle,\left| E_2 \right\rangle \right)$ are
        \begin{equation*}
            \begin{aligned}
                c_1 & = \left\langle E_1 | \downarrow \right\rangle = \frac{1}{\sqrt{2}}\left( \left\langle \uparrow \right|+\left\langle \downarrow \right| \right) \left| \downarrow \right\rangle = \frac{1}{\sqrt{2}} \\
                c_2 & = \left\langle E_2 | \downarrow \right\rangle = \frac{1}{\sqrt{2}}\left( \left\langle \uparrow \right|-\left\langle \downarrow \right| \right) \left| \downarrow \right\rangle = - \frac{1}{\sqrt{2}}
            \end{aligned} ,
        \end{equation*}
        which means
        \begin{equation*}
            \left| \psi\left( 0 \right) \right\rangle = \frac{1}{\sqrt{2}} \left| E_1 \right\rangle - \frac{1}{\sqrt{2}}\left| E_2 \right\rangle.
        \end{equation*}
        It follows that
        \begin{equation*}
            \begin{aligned}
                \left| \psi\left( t \right) \right\rangle & = U\left( t,0 \right)\left| \psi\left( 0 \right) \right\rangle 
                = \left( \sum^{}_{n} e^{-\frac{i}{\hbar}E_nt} \left| E_n \right\rangle \left\langle E_n \right| \right) \left| \psi\left( 0 \right) \right\rangle
                = \left( e^{-i\Omega t}\left| E_1 \right\rangle \left\langle E_1 \right| + e^{i\Omega t} \left| E_2 \right\rangle \left\langle E_2 \right| \right) \frac{1}{\sqrt{2}} \left( \left| E_1 \right\rangle-\left| E_2 \right\rangle \right)  \\
                                                          & = \frac{1}{\sqrt{2}}e^{-i\Omega t}\left| E_1 \right\rangle - \frac{1}{\sqrt{2}} e^{i\Omega t} \left| E_2 \right\rangle
                                                          = \frac{1}{2} e^{-i\Omega t} \left( \left| \uparrow \right\rangle+\left| \downarrow \right\rangle \right) - \frac{1}{2}e^{i\Omega t} \left( \left| \uparrow \right\rangle-\left| \downarrow \right\rangle \right) 
                                                          = \frac{1}{2} \left( e^{-i\Omega t}-e^{i\Omega t} \right)\left| \uparrow \right\rangle + \frac{1}{2} \left( e^{-\Omega t}+e^{i\Omega t} \right)\left| \downarrow \right\rangle \\
                                                          & = i\sin\left( \Omega t \right)\left| \uparrow \right\rangle + \cos\left( \Omega t \right)\left| \downarrow \right\rangle.
            \end{aligned} 
        \end{equation*}
    \end{example}
    
    \rruleline

    \subsection{Harmonic Oscillator}

    \np Consider the Hamiltonian operator
    \begin{equation}
        H = \frac{p^{2}}{2m} = \frac{1}{2} m\omega^{2} x^{2}.
    \end{equation}
    Suppose $H\left| \psi \right\rangle = E\left| \psi \right\rangle$, where $E$ is an eigenvalue of $H$. How do we calculate $E,\left| \psi \right\rangle$?

    \np First approach: substitute
    \begin{equation}
        p = -i\hbar \frac{\partial}{\partial x}
    \end{equation}
    into [1.1]. We won't use this approach, as it does not give much intuition to the problem.

    \np Second approach: rewrite Hamiltonian in terms of \textit{ladder operators}. Let
    \begin{equation*}
        a = \frac{1}{\sqrt{2\hbar m\omega}} \left( ip + m\omega x \right),
    \end{equation*}
    so that
    \begin{equation*}
        a^{\dagger} = \frac{1}{\sqrt{2\hbar m\omega}}\left( -ip+m\omega x \right).
    \end{equation*}
    Then, using the commutator formula $\left[ x,p \right] = i\hbar$,
    \begin{equation*}
        \begin{aligned}
            a^{\dagger}a & = \frac{1}{2\hbar m\omega} \left( -ip+m\omega x \right)\left( ip+m\omega x \right) = \frac{1}{2\hbar m\omega} \left( p^{2}+im\omega\left( xp-px \right)+m^{2}\omega^{2}x^{2} \right) \\
                         & = \frac{1}{2\hbar m\omega} \left( p^{2}+im\omega\left[ x,p \right]+m^{2}\omega^{2}x^{2} \right) = \frac{1}{2\hbar m\omega} \left( p^{2}+im\omega i\hbar+m^{2}\omega^{2}x^{2} \right) \\
                         & = \frac{1}{2\hbar m\omega} \left( p^{2}+m^{2}\omega^{2}x^{2}-\hbar m\omega \right) = \frac{1}{\hbar\omega} \left( \frac{p^{2}}{2m} + \frac{1}{2} m\omega^{2}x^{2} \right) - \frac{1}{2} = \frac{1}{\hbar\omega} H - \frac{1}{2}.
        \end{aligned} 
    \end{equation*}
    Thus we conclude
    \begin{equation}
        H = \hbar\omega\left( a^{\dagger}a+\frac{1}{2} \right).
    \end{equation}
    Observe that [1.3] tells us that the eigenvectors of $H$ are precisely the eigenvectors of $a^{\dagger}a$; 
    \begin{equation*}
        a^{\dagger}a\left| E_n \right\rangle = n\left| E_n \right\rangle \implies H\left| E_n \right\rangle = \hbar\omega\left( n+\frac{1}{2} \right)\left| E_n \right\rangle.
    \end{equation*}

    \np We can find $n$ using commutators. First
    \begin{equation*}
        \begin{aligned}
            \left[ a,a^{\dagger} \right] & = \frac{1}{2\hbar m\omega} \left[ ip+m\omega x, -ip+m\omega x \right] = \frac{1}{2\hbar m\omega} \left( \left[ ip,-ip \right] + \left[ ip,m\omega x \right] + \left[ m\omega x,-ip \right] + \left[ m\omega x, m\omega x \right]\right) \\
                                         & = \frac{1}{2\hbar m\omega} \left( im\omega\left[ p,x \right] -im\omega \left[ x,p \right] \right) = \frac{i}{2\hbar} \left( -i\hbar-i\hbar \right) = 1.
        \end{aligned} 
    \end{equation*}
Now, observe that $a\left| E_n \right\rangle$ is an eigenstate of $a^{\dagger}a$ corresponding to $n-1$: since $\left[ a,a^{\dagger} \right]=1$,
    \begin{equation*}
        a^{\dagger}a\left( a\left| E_n \right\rangle \right) = \left( aa^{\dagger}-1 \right)a\left| E_n \right\rangle = a\left( a^{\dagger}a \right)\left| E_n \right\rangle - a\left| E_n \right\rangle = an\left| E_n \right\rangle - a\left| E_n \right\rangle = \left( n-1 \right)a\left| E_n \right\rangle.
    \end{equation*}
    Hence
    \begin{equation*}
        a\left| E_n \right\rangle = c_n\left| E_{n-1} \right\rangle.
    \end{equation*}
    Similarly,
    \begin{equation*}
        a^{\dagger}\left| E_n \right\rangle = b_n\left| E_{n+1} \right\rangle.
    \end{equation*}
    
    Let us find what $c_n,b_n$ are. Observe
    \begin{equation*}
        n = \left\langle E_n \right|n\left| E_n \right\rangle = \left\langle E_n \right|a^{\dagger}a\left| E_n \right\rangle = \left\langle E_{n-1} \right| c_n^{*}c_n \left| E_{n-1} \right\rangle = \left| c_n \right|^{2}.
    \end{equation*}
    Therefore $n$ is nonnegative and
    \begin{equation*}
        c_n = \sqrt{n}
    \end{equation*}
    or
    \begin{equation*}
        a\left| E_n \right\rangle = \sqrt{n}\left| E_{n-1} \right\rangle.
    \end{equation*}

    In a similar fashion, we find out
    \begin{equation*}
        b_n = \sqrt{n+1},
    \end{equation*}
    so that
    \begin{equation*}
        a^{\dagger}\left| E_n \right\rangle = \sqrt{n+1}\left| E_{n+1} \right\rangle.
    \end{equation*}

    \np Note that we have to have $n$ to be an integer, so that
    \begin{equation*}
        a^n\left| E_n \right\rangle = 0.
    \end{equation*}

    \np Since we may write $x,p$ in terms of the ladder operators as
    \begin{equation*}
        x = \sqrt{\frac{\hbar}{2m\omega}}\left( a+a^{\dagger} \right)
    \end{equation*}
    and
    \begin{equation*}
        p = i\sqrt{\frac{\hbar m\omega}{2}}\left( a^{\dagger}-a \right),
    \end{equation*}
    we have
    \begin{equation*}
        \left< x^{2} \right> = \left\langle E_n \right|x^{2}\left| E_n \right\rangle = \frac{\hbar}{2m\omega} \left\langle E_n \right|aa+aa^{\dagger}+a^{\dagger}a+a^{\dagger}a^{\dagger}\left| E_n \right\rangle = \frac{\hbar}{2m\omega} \left\langle E_n \right|aa+2a^{\dagger}a+1+a^{\dagger}a^{\dagger}\left| E_n \right\rangle.
    \end{equation*}
    But, using orthonormality of $\left\lbrace \left| E_n \right\rangle \right\rbrace^{\infty}_{n=0}$, we obtain
    \begin{equation*}
        \left\langle E_n \right|aa\left| E_n \right\rangle = \left\langle E_n \right|a^{\dagger}a^{\dagger}\left| E_n \right\rangle = 0,
    \end{equation*}
    so that
    \begin{equation*}
        \left< x^{2} \right> = \frac{\hbar}{2m\omega}\left\langle E_n \right|2a^{\dagger}a+1\left| E_n \right\rangle = \frac{\hbar}{2m\omega} \left( 2n+1 \right) \left\langle E_n | E_n \right\rangle = \frac{\hbar}{2m\omega} \left( 2n+1 \right).
    \end{equation*}

    \begin{summary}{Harmonic Oscillator, Ladder Operators}
        Given Hamiltonian
        \begin{equation*}
            H = \frac{1}{2}m\omega^{2}x^{2},
        \end{equation*}
        we have that
        \begin{equation*}
            H = \hbar\omega\left( a^{\dagger}a+\frac{1}{2} \right),
        \end{equation*}
        where
        \begin{equation*}
            a = \frac{1}{\sqrt{2\hbar m\omega}}\left( ip+m\omega x \right)
        \end{equation*}
        is the \textit{ladder operator}. $a,a^{\dagger}$ acts on the energy eigenstates by
        \begin{equation*}
            a\left| E_n \right\rangle = \sqrt{n}\left| E_{n-1} \right\rangle
        \end{equation*}
        and
        \begin{equation*}
            a^{\dagger}\left| E_n \right\rangle = \sqrt{n+1}\left| E_{n+1} \right\rangle .
        \end{equation*}
        With respect to the ladder operators, $x,p$ can be written as
        \begin{equation*}
            x = \sqrt{\frac{\hbar}{2m\omega}}\left( a+a^{\dagger} \right)
        \end{equation*}
        and
        \begin{equation*}
            p = i\sqrt{\frac{\hbar m\omega}{2}}\left( a^{\dagger}-a \right),
        \end{equation*}
    \end{summary}

    \begin{example}{}
        Consider the harmonic oscillator states $\left| 0 \right\rangle = \left| E_0 \right\rangle, \ldots$ and suppose $\left| \phi\left( 0 \right) \right\rangle = \frac{1}{\sqrt{2}}\left( \left| n \right\rangle+\left| n+1 \right\rangle \right)$. What is $\left< x\left( t \right) \right>$? 
    \end{example}

    \begin{answer}
        Observe that
        \begin{equation*}
            \left| \phi\left( t \right) \right\rangle = \frac{1}{\sqrt{2}} \left( e^{-\frac{i}{\hbar}E_nt}\left| n \right\rangle+e^{-\frac{i}{\hbar}E_{n+1}t}\left| n+1 \right\rangle \right) 
            = \frac{1}{\sqrt{2}} \left( e^{-i\omega\left( n+\frac{1}{2} \right)t}\left| n \right\rangle+e^{-i\omega\left( n+\frac{3}{2} \right)t}\left| n+1 \right\rangle \right) ,
        \end{equation*}
        so that
        \begin{equation*}
            \begin{aligned}
                \left< x\left( t \right) \right> & = \left\langle \psi\left( t \right) \right|x\left| \psi\left( t \right) \right\rangle = \sqrt{\frac{\hbar}{2m\omega}}\left\langle \psi\left( t \right) \right| a+a^{\dagger} \left| \psi\left( t \right) \right\rangle \\
                & = \sqrt{\frac{\hbar}{2m\omega}} \frac{1}{2} \left( \left\langle n \right|e^{i\omega\left( n+\frac{1}{2} \right)t}+\left\langle n+1 \right|e^{i\omega\left( n+\frac{3}{2} \right)t} \right) 
                        \left( a+a^{\dagger} \right)\sqrt{\frac{\hbar}{2m\omega}} \frac{1}{2} \left( e^{-i\omega\left( n+\frac{1}{2} \right)t}\left| n \right\rangle+e^{-i\omega\left( n+\frac{3}{2} \right)t}\left| n+1 \right\rangle \right)  \\
                & = \frac{1}{2}\sqrt{\frac{\hbar}{2m\omega}} \left(e^{i\omega+\left( n+\frac{3}{2} \right)t}  e^{-i\omega \left( n+\frac{1}{2} \right)t} \left\langle n+1 \right|a^{\dagger}\left| n \right\rangle + e^{i\omega\left( n+\frac{1}{2} \right)t}e^{-i\omega\left( n+\frac{3}{2} \right)t} \left\langle n \right|a\left| n+1 \right\rangle\right) \\
                & = \frac{1}{2} \sqrt{\frac{\hbar}{2m\omega}} \left( e^{i\omega t} \sqrt{n+1} + e^{-i\omega t} \sqrt{n+1} \right) = \sqrt{\frac{\hbar\left( n+1 \right)}{2m\omega}} \cos\left( \omega t \right).
            \end{aligned} 
        \end{equation*}
        Thus
        \begin{equation*}
            \left< x\left( t \right) \right> = \sqrt{\frac{\hbar\left( n+1 \right)}{2m\omega}}\cos\left( \omega t \right).
        \end{equation*}
    \end{answer}

    \subsection{Angular Momentum Commutators}

    We start with classical \textit{orbital angular momentum} (OAM):
    \begin{equation*}
        \vec{L} = \vec{r}\times\vec{p}.
    \end{equation*}
    By definition,
    \begin{equation*}
        \vec{L} = \begin{bmatrix} ypz-zp_y \\ zp_x-xp_z \\ xp_y-yp_x \end{bmatrix}.
    \end{equation*}
    We are going to use \textit{canonical commutation relations}
    \begin{equation*}
        \begin{aligned}
            \left[ r_i,r_j \right] & = 0 \\
            \left[ p_i,p_j \right] & = 0 \\
            \left[ r_i,p_j \right] & = i\hbar\delta_{i,j}
        \end{aligned} ,
    \end{equation*}
    where $r_1,r_2,r_3$ are position components (say $x,y,z$ for instance) and $p_1,p_2,p_3$ are the momentum components, to figure out $\left[ L_i,L_j \right]$. For instance,
    \begin{equation*}
        \left[ L_x,L_y \right] = \left[ yp_z-zp_y,zp_x-xp_z \right] = \left[ yp_z,zp_x \right]-\left[ zp_y,zp_x \right]-\left[ yp_z,xp_z \right]+\left[ zp_y,xp_z \right].
    \end{equation*}
    But note that $z$ commutes with itself and $p_y,p_z$ commutes, so that $\left[ zp_y,zp_x \right]=0$. Similarly, $x,y$ commutes so that $\left[ yp_z,xp_z \right]=0$. Hence,
    \begin{equation*}
        \left[ L_x,L_y \right] = \left[ yp_z,zp_x \right] + \left[ zp_y,xp_z \right] = y\left[ p_z,zp_x \right]+\left[ y,zp_x \right]p_z + \left[ zp_y,xp_z \right].
    \end{equation*}
    Since $y$ commutes with $z,p_x$, it follows $\left[ y,zp_x \right]=0$. This means
    \begin{equation*}
        \left[ L_x,L_y \right] = y\left[ p_z,zp_x \right]+\left[ zp_y,xp_z \right] = yz\underbrace{\left[ p_z,p_x \right]}_{=0} + y\left[ p_z,z \right]p_x+xz\underbrace{\left[ p_y,p_z \right]}_{=0}+x\left[ z,p_z \right]p_y = i\hbar\left( xp_y-yp_x \right) = i\hbar L_z.
    \end{equation*}
    Similarly, we have
    \begin{equation*}
        \left[ L_y,L_z \right] = i\hbar L_x, \left[ L_z,L_x \right] = i\hbar L_y.
    \end{equation*}

    \begin{summary}{Orbital Angular Momentum Commutation Relation}
        Let $L = \left( L_x,L_y,L_z \right)$ be the orbital angular momentum operator. Then
        \begin{equation*}
            \left[ L_x,L_y \right] = i\hbar L_z, \left[ L_y,L_z \right] = i\hbar L_x, \left[ L_z,L_x \right] = i\hbar L_y.
        \end{equation*}
    \end{summary}

    \np We are going to \textit{define} angular momentum operator $J$ as $J = \left( J_x,J_y,J_z \right)$ satisfying the above relations. That is,
    \begin{equation*}
        \left[ J_i,J_j \right] = i\hbar J_k \epsilon_{ijk}, \hspace{2cm}\forall i,j,k.
    \end{equation*}
    $\epsilon_{ijk}$ is the \textit{Levi-Civita symbol}.

    Along with
    \begin{equation*}
        J^{2} = J_x^{2}+J_y^{2}+J_z^{2},
    \end{equation*}
    we can find quantization restrictions on amount of angular momentums.
    
    \begin{claim}
        $\left[ J^{2},J_z \right] = 0$.

        Observe that
        \begin{equation*}
            \begin{aligned}
                \left[ J^{2},J_z \right] & = \left[ J_x^{2}+J_y^{2}+J_z^{2},J_z \right] = \left[ J_x^{2},J_z \right] + \left[ J_y^{2},J_z \right] + \underbrace{\left[ J_z^{2},J_z \right]}_{=0} \\
                                         & = J_x\underbrace{\left[ J_x,J_z \right]}_{=-i\hbar L_y} + \left[ J_x,J_z \right]J_x + J_y\underbrace{\left[ J_y,J_z \right]}_{=i\hbar J_x} + \left[ J_y,J_z \right]J_y= -i\hbar J_xJ_y - i\hbar J_yJ_x + i\hbar J_yJ_x + i\hbar J_xJ_y = 0.
            \end{aligned} 
        \end{equation*}
    \end{claim}

    \np In a similar manner, we obtain that
    \begin{equation*}
        \left[ J^{2},J_x \right] = \left[ J^{2},J_y \right] = \left[ J^{2},J_z \right] = 0.
    \end{equation*}
    Hence we can find a common eigenbasis for $J^{2},J_i$ for all $i$.

    \np We are also going to introduce \textit{ladder operators} for angular momentum (these operators are not the same as in quantum harmonic oscillators):
    \begin{equation*}
        J_{\pm} = J_x \pm iJ_y.
    \end{equation*}
    Observe that
    \begin{equation*}
        \left[ J^{2},J_{\pm} \right] = 0,
    \end{equation*}
    since $J_{\pm}$ is a linear combination of $J_x,J_y$ which commute with $J^{2}$. Also,
    \begin{equation*}
        \left[ J_z,J_{\pm} \right] = \left[ J_z,J_x \right]\pm i\left[ J_z,J_y \right] = i\hbar J_y \pm i\left( -i\hbar J_x \right) = i\hbar J_y\pm\hbar J_x = \pm\hbar\left( J_x\pm iJ_y \right) = \pm\hbar J_{\pm}.
    \end{equation*}

    \np Let us find the common eigenbasis for $J^{2},J_z$ first. Consider the system of eigenvalue equations
    \begin{equation*}
        \begin{aligned}
            J^{2}\left| a,b \right\rangle & = a\left| a,b \right\rangle \\
            J_z\left| a,b \right\rangle & = b\left| a,b \right\rangle
        \end{aligned} .
    \end{equation*}
    Similar to how we did to harmonic oscillators, apply the raising operator $J_+$ to $\left| a,b \right\rangle$. Observe that
    \begin{equation*}
        J_z\left( J_+\left| a,b \right\rangle \right) = J_zJ_+\left| a,b \right\rangle = \left( J_+J_z+\hbar J_+ \right)\left| a,b \right\rangle = J_+b\left| a,b \right\rangle + \hbar J_+\left| a,b \right\rangle = \left( b+\hbar \right)J_+\left| a,b \right\rangle.
    \end{equation*}
    That is, $J_+\left| a,b \right\rangle$ is an eigenvector of $J_z$ corresponding to $b+\hbar$, so we may write
    \begin{equation*}
        J_+\left| a,b \right\rangle = c\left| a',b+\hbar \right\rangle
    \end{equation*}
    for some $a'$. In a similar manner, we have
    \begin{equation*}
        J^{2}J_+ = aJ_+\left| a,b \right\rangle,
    \end{equation*}
    so that $J_+\left| a,b \right\rangle$ is an eigenvector of $J^{2}$ corresponding to $a$, which means
    \begin{equation}
        J_{+}\left| a,b \right\rangle = c\left| a,b+\hbar \right\rangle.
    \end{equation}
    We can also get:
    \begin{equation}
        J_-\left| a,b \right\rangle = c'\left| a,b-\hbar \right\rangle.
    \end{equation}
    
    \np Let us utilize [1.4], [1.5] to find which values of $a,b$ are allowed. 

    \begin{claim}
        $\left< J^{2}-J_z^{2} \right>_{\left| a,b \right\rangle}\geq 0$. 

        Since $\left| a,b \right\rangle$ is an eigenvector of $J^{2},J_z$ corresponding to $a,b$, respectively,
        \begin{equation*}
            \left< J^{2}-J_z^{2} \right>_{\left| a,b \right\rangle} = a-b^{2}. 
        \end{equation*}
        We also have
        \begin{equation*}
            J^{2}-J_z^{2} = J_x^{2}+J_y^{2}+J_z^{2}-J_z^{2} = J_x^{2}+J_y^{2} = \frac{1}{2} \left( J_+J_-+J_-J_+ \right),
        \end{equation*}
        so that
        \begin{equation*}
            \left< J^{2}-J_z^{2} \right> = \left< \frac{1}{2}\left( J_+J_-+J_-J_+ \right) \right> = \frac{1}{2} \left< J_+J_-+J_-J_+ \right>. 
        \end{equation*}
        Since $J_{\pm} = J_x\pm iJ_y$ where $J_x,J_y$ are Hermitian, it follows that $J_{\pm}^{\dagger} = J_{\mp}$. Therefore, $J_+J_-,J_-J_+$ are Hermitian. This means
        \begin{equation*}
            \left< J_+J_-+J_-J_+ \right> = \left< J_+J_+^{\dagger}+J_+^{\dagger}J_+ \right> = \left< J_+J_+^{\dagger} \right> + \left< J_+^{\dagger}J_+ \right> = \left\lVert J_+^{\dagger}\left| a,b \right\rangle\right\rVert + \left\lVert J_+\left| a,b \right\rangle\right\rVert \geq 0.
        \end{equation*}
    \end{claim}

    \np Consequently,
    \begin{equation*}
        a-b^{2}\geq 0 \implies a\geq b^{2}.
    \end{equation*}
    Because of this define $b_{+}$ be the smallest value such that
    \begin{equation*}
        \left( b_{+}+\hbar \right)^{2}>a.
    \end{equation*}
    Using the properties of raising and lowering operators, we can write the value of $b_{+}$ in terms of $a$. For this $b_{+}$, we have
    \begin{equation*}
        J_+\left| a,b_{+} \right\rangle = 0.
    \end{equation*}
    Similarly, let $b_{-}$ be the largest value such that $\left( b_--\hbar \right)^{2}>a$, so that
    \begin{equation*}
        J_-\left| a,b_{-} \right\rangle = 0.
    \end{equation*}
    Then
    \begin{equation*}
        J_-J_+\left| a,b_{+} \right\rangle = 0,
    \end{equation*}
    where
    \begin{equation*}
        J_-J_+ = \left( J_x-iJ_y \right)\left( J_x+iJ_y \right) = \underbrace{J_x^{2}+J_y^{2}}_{=J^{2}-J_z^{2}}+i\underbrace{\left( J_xJ_y-J_yJ_x \right)}_{=\left[ J_x,J_y \right]=i\hbar J_z} = J^{2}-J_z^{2}-\hbar J_z.
    \end{equation*}
    This means
    \begin{equation*}
        0 = J_-J_+\left| a,b_{+} \right\rangle = \left( J^{2}-J_z^{2}-\hbar J_z \right)\left| a,b_+ \right\rangle = \left( a-b_+^{2}-\hbar b_+ \right)\left| a,b_+ \right\rangle,
    \end{equation*}
    solving which gives
    \begin{equation}
        a = b_+\left( b_++\hbar \right).
    \end{equation}
    In a similar manner, we obtain
    \begin{equation}
        a = b_-\left( b_--\hbar \right).
    \end{equation}
    Note [1.6], [1.7] are satisfied if $b_- = -b_+$.

    Assuming that \textit{we can obtain $b_+$ by applying $J_+$ $n\in\N$ times on $b_-$}, we gain
    \begin{equation*}
        b_+ = b_-+n\hbar = -b_++n\hbar \implies b_+ = \frac{n\hbar}{2}.
    \end{equation*}
    It follows that
    \begin{equation}
        a = \frac{n\hbar}{2}\left( \frac{n\hbar}{2}+\hbar \right) = \hbar^{2} \frac{n}{2} \left( \frac{n}{2}+1 \right).
    \end{equation}

    Since the expression $\hbar \frac{n}{2}\left( \frac{n}{2}+1 \right)$ in [1.8] is \textit{ugly}, we substitute $j = \frac{n}{2}, m = \frac{b}{\hbar}$ so that
    \begin{equation}
        \begin{aligned}
            J^{2}\left| j,m \right\rangle & = \hbar^{2}j\left( j+1 \right)\left| j,m \right\rangle \\
            J_z\left| j,m \right\rangle & = \hbar m\left| j,m \right\rangle
        \end{aligned} ,
    \end{equation}
    where $j$ is a half-integer and $m$ satisfy $m\in\left\lbrace -j,-j+1,\ldots,j-1,j \right\rbrace$.

    [1.9] is the usual convention for angular momentum operators.

    \begin{summary}{Angular Momentum Operators}
        We define angular momentum operators $J = \left( J_x,J_y,J_z \right)$ as operators satisfying
        \begin{equation*}
            \left[ J_i,J_j \right] = i\hbar J_k\epsilon_{ijk}, \hspace{2cm}\forall i,j,k.
        \end{equation*}
        We define $J^{2} = J_x^{2}+J_y^{2}+J_z^{2}$, which commutes with $J_x,J_y,J_z$:
        \begin{equation*}
            \left[ J^{2},J_x \right] = \left[ J^{2},J_y \right] = \left[ J^{2},J_z \right] = 0.
        \end{equation*}

        The ladder operators are
        \begin{equation*}
            J_+ = J_x+iJ_y, J_- = J_x-iJ_y,
        \end{equation*}
        which are also called the raising and lowering operators. Then
        \begin{equation*}
            \left[ J_z,J_{\pm} \right] = \pm\hbar J_{\pm}.
        \end{equation*}

        The eigenvalues of $J^{2},J_z$ are
        \begin{equation*}
            \begin{aligned}
                J^{2}\left| j,m \right\rangle & = \hbar^{2}j\left( j+1 \right)\left| j,m \right\rangle \\
                J_z\left| j,m \right\rangle & = \hbar m\left| j,m \right\rangle
            \end{aligned} ,
        \end{equation*}
        where $j$ is a half-integer and $m\in\left\lbrace -j,-j+1,\ldots,j \right\rbrace$. 
    \end{summary}

    \np Notationally, we write $J = L$ and $j=l$ for orbital angular momentum and $J=S$ and $j=s$ for spin angular momentum.

    \np In case $J = L$, an orbital angular momentum, the only allowed values of $j=l$ are integers. But in case $J=S$, a spin angular momentum, any half integer $j=s$ is allowed.

    \np Let us find the matrix representation of the operators. Since
    \begin{equation*}
        J_{\pm} \left| j,m \right\rangle = c_{\pm}\left| j,m\pm 1 \right\rangle
    \end{equation*}
    for some constants $c_{\pm}$,
    \begin{equation*}
        \left\langle j,m \right|J_-J_+\left| j,m \right\rangle = \left| c_+ \right|^{2} \left\langle j,m | j,m \right\rangle = \left| c_+ \right|^{2}.
    \end{equation*}
    But we also know that
    \begin{equation*}
        J_-J_+ = \cdots = J^{2}-J_z^{2}-\hbar J_z,
    \end{equation*}
    so that
    \begin{equation*}
        \left| c_+ \right|^{2} = \left\langle j,m \right|J_-J_+\left| j,m \right\rangle = \hbar^{2}j\left( j+1 \right)-\left( \hbar m \right)^{2}-\hbar^{2}m = \hbar^{2}\left( j\left( j+1 \right)-m\left( m+1 \right) \right).
    \end{equation*}
    Ignoring global phase, it follows
    \begin{equation*}
        c_+ = \hbar\sqrt{j\left( j+1 \right)-m\left( m+1 \right)},
    \end{equation*}
    so that
    \begin{equation*}
        J_+\left| j,m \right\rangle = \hbar\sqrt{j\left( j+1 \right)-m\left( m+1 \right)}.
    \end{equation*}
    In a similar manner, we find
    \begin{equation*}
        J_-\left| j,m \right\rangle = \hbar\sqrt{j\left( j+1 \right)-m\left( m-1 \right)}.
    \end{equation*}
    We can use this to find matrix representations.

    \begin{example}{}
        Suppose $j=\frac{1}{2}$, so that we have two eigenstates
        \begin{equation*}
            \begin{aligned}
                \left| 1 \right\rangle & = \left| \frac{1}{2},\frac{1}{2} \right\rangle \\
                \left| 0 \right\rangle & = \left| \frac{1}{2},-\frac{1}{2} \right\rangle
            \end{aligned} .
        \end{equation*}
        Then $J_z\left| 1 \right\rangle = \frac{\hbar}{2}\left| 1 \right\rangle, J_z\left| 0 \right\rangle = \frac{\hbar}{2}\left| 0 \right\rangle$, so that
        \begin{equation*}
            J_z = \frac{\hbar}{2} \begin{bmatrix} 1 & 0 \\ 0 & -1 \end{bmatrix}
        \end{equation*}
        is the matrix representation of $J_z$.

        For $J_x$, recall
        \begin{equation*}
            J_x = \frac{1}{2}\left( J_+-J_- \right).
        \end{equation*}
        The matrix representation of $J_+$ is
        \begin{equation*}
            J_+ =
            \begin{bmatrix}
            	0 & \hbar \\
            	0 & 0 \\
            \end{bmatrix}.
        \end{equation*}
        On the other hand,
        \begin{equation*}
            J_- = 
            \begin{bmatrix}
            	0 & 0 \\
            	\hbar & 0 \\
            \end{bmatrix},
        \end{equation*}
        so that
        \begin{equation*}
            J_x = \frac{\hbar}{2} 
            \begin{bmatrix}
            	0 & 1 \\
            	1 & 0 \\
            \end{bmatrix}.
        \end{equation*}

        In a similar manner,
        \begin{equation*}
            J_y = -\frac{i}{2}\left( J_+-J_- \right) = \frac{\hbar}{2}
            \begin{bmatrix}
            	0 & -i \\
            	i & 0 \\
            \end{bmatrix}
        \end{equation*}

        Squaring each $J_x,J_y,J_z$, we obtain
        \begin{equation*}
            J^{2} = \frac{3}{4}\hbar^{2}
            \begin{bmatrix}
            	1 & 0 \\
            	0 & 1 \\
            \end{bmatrix}.
        \end{equation*}
    \end{example}
    
    \rruleline

    \subsection{Spherical Harmonics}

    \np Consider $\vec{L}=\vec{r}\times\vec{p}$. We expect to see
    \begin{equation*}
        \begin{aligned}
            J^{2}\left| j,m \right\rangle & = \hbar^{2}j\left( j+1 \right)\left| j,m \right\rangle \\
            J_z\left| j,m \right\rangle & = \hbar m\left| j,m \right\rangle
        \end{aligned} .
    \end{equation*}
    Moreover, we want an explicit form of $\left| l,m \right\rangle$ in 3D coorinates. To do so, we can use
    \begin{equation*}
        L_z = -i\hbar x\partial_y + i\hbar y\partial_x
    \end{equation*}
    or use \textit{spherical coordinates}. That is
    \begin{equation*}
        \vec{L} = \vec{r}\times\vec{p} = -i\hbar\vec{r}\times\vec{\nabla},
    \end{equation*}
    where
    \begin{equation*}
        \vec{\nabla} = \hat{r}\partial_r + \frac{1}{r}\hat{\theta} \partial_{\theta} + \frac{1}{r\sin\left( \theta \right)} \hat{\phi}\partial_{\phi} .
    \end{equation*}
    Then
    \begin{equation*}
        \begin{aligned}
            L^{2} & = -\hbar^{2} \left( \frac{\partial^{2}}{\partial\theta^{2}} + \cot\left( \theta \right) \frac{\partial}{\partial\theta} + \frac{1}{\sin^{2}\left( \theta \right)} \frac{\partial^{2}}{\partial\phi^{2}} \right) \\
            L_x & = -i\hbar\left( -\sin\left( \phi \right)\partial_{\theta} - \cot\left( \theta \right)\cos\left( \phi \right)\partial_{\phi} \right) \\
            L_y & = -i\hbar\left( \cos\left( \phi \right)\partial_{\theta} - \cot\left( \theta \right)\sin\left( \phi \right)\partial_{\phi} \right) \\
            L_z & = -i\hbar\partial_{\phi}
        \end{aligned}. 
    \end{equation*}
    Observe that $L^{2},L_x,L_y,L_z$ \textit{does not depend} on the radial coordinate $r$. Hence an eigenfunction of $L^{2},L_z$ depends only on $\theta,\phi$, say
    \begin{equation*}
        \begin{aligned}
            L^{2}Y\left( \theta,\phi \right) & = \hbar^{2}l\left( l+1 \right)Y\left( \theta,\phi \right) \\
            L_zY\left( \theta,\phi \right) & = \hbar mY\left( \theta,\phi \right)
        \end{aligned} .
    \end{equation*}

    As an \textit{ansatz}, suppose
    \begin{equation*}
        Y\left( \theta,\phi \right) = \Theta\left( \theta \right)\Phi\left( \phi \right).
    \end{equation*}
    In this case,
    \begin{equation*}
        L_zY\left( \theta,\phi \right) = \hbar mY\left( \theta,\phi \right)
    \end{equation*}
    becomes a differential equation
    \begin{equation*}
        -i\hbar\partial_{\phi} \Theta\Phi = \hbar m\Theta\Phi.
    \end{equation*}
    But $\Theta$ only depends on $\theta$, so that
    \begin{equation*}
        -i\hbar\partial_{\phi}\Phi = \hbar m\Phi \implies \Phi\left( \theta \right) = e^{im\phi}.
    \end{equation*}
    By the $2\pi$ periodicy of $e^{i\theta}$, we have
    \begin{equation}
        e^{im\phi} = e^{im\phi}e^{2\pi mi}.
    \end{equation}
    Observe that [1.10] is satisfied if and only if $m$ is an integer. This is why for orbital angular momentums integer values of $m$ are only permitted values.

    Now that we know how $\Phi$ looks like, we have
    \begin{equation*}
        L^{2}Y\left( \theta,\phi \right) = L^{2}e^{im\phi}\Theta\left( \theta \right). = \hbar^{2} l\left( l+1 \right)e^{im\phi}\Theta\left( \theta \right)
    \end{equation*}
    Recall that
    \begin{equation*}
        L^{2} = -\hbar^{2}\left( \partial_{\theta}^{2}+\cot\left( \theta \right)\partial_{\theta}+\frac{1}{\sin^{2}\left( \theta \right)}\partial_{\phi}^{2} \right) ,
    \end{equation*}
    so that we have
    \begin{equation*}
        -\hbar^{2}\left( \partial_{\theta}^{2}+\cot\left( \theta \right)\partial_{\theta}+\frac{1}{\sin^{2}\left( \theta \right)}\partial_{\phi}^{2} \right) e^{im\phi}\Theta\left( \theta \right) = \hbar^{2} l\left( l+1 \right)\Phi\left( \theta \right).
    \end{equation*}
    Cancelling out $\hbar^{2},e^{im\phi}$,
    \begin{equation*}
        -\partial_{\theta}^{2}\Theta-\cot\left( \theta \right)\partial_{\theta}\Theta-\frac{1}{\sin^{2}\left( \theta \right)}\partial_{\phi}^{2} \left( -m^{2} \right)\Theta = l\left( l+1 \right)\Theta,
    \end{equation*}
    or
    \begin{equation}
        \left(-\partial_{\theta}^{2}-\cot\left( \theta \right)\partial_{\theta}+\frac{m^{2}}{\sin^{2}\left( \theta \right)}\partial_{\phi}^{2} - l\left( l+1 \right)\right) \Theta = 0,
    \end{equation}
    which is of the form of the \textit{general Legendre equation}. A general solution to [1.11] is
    \begin{equation*}
        \Theta = P_l^m\left(\cos\left( \theta \right)\right),
    \end{equation*}
    where the \textit{Legendre polynomial} $P_l^m$ is
    \begin{equation*}
        P_l^m = \left( 1-x^{2} \right)^{\frac{\left| m \right|}{2}} \left( \frac{d}{dx} \right)^{\left| m \right|} P_l
    \end{equation*}
    and
    \begin{equation}
        P_l = \frac{1}{2^ll!}\left( \frac{d}{dx} \right)^l \left( x^{2}-1 \right)^l.
    \end{equation}
    Since [1.12] is \textit{usually} defined for nonnegative integer $l$ only, it puts additional restriction on $l$: $l\in\N\cup\left\lbrace 0 \right\rbrace$.

    In conclusion, the spherical harmonics $Y_l^m\left( \theta,\phi \right)$ have the form
    \begin{equation*}
        Y_l^m = A P_l^m\left( \cos\left( \theta \right) \right)e^{im\phi},
    \end{equation*}
    where $A$ is a normalization constant. $Y_l^m\left( \theta,\phi \right)$ are the eigenstates of $L^{2},L_z$ with eigenvalues $\hbar^{2}l\left( l+1 \right),\hbar m$. The collection $\left\lbrace Y_l^m \right\rbrace$ is an orthonormal basis for the space of functions of $\theta,\phi$.

    \clearpage

    -- LECTURE 15 --

    \clearpage
    
    \np Recall that symmetric 2D/3D systems feature degeneracy (i.e. there are different states with same energy). Hence we need another observable to \textit{unambiguously} label eigenstates. That is, we are going to pick an observable $A$ with
    \begin{equation*}
        \left[ A,H \right] = 0
    \end{equation*}
    and use eigenvalues of $A$ as a second \textit{label}.

    \np We are going to consider \textit{complete set of commuting operators}, CSCO: all operators commute with each other and eigenvalues of all operators are sufficient to fully distinguish basis states. Recall that
    \begin{equation*}
        H = \frac{p_x^{2}+p_y^{2}}{2m} + \frac{m\omega^{2}}{2}\left( x^{2}+y^{2} \right) = \hbar\omega\left( a^{\dagger}_xa_x + a_y^{\dagger}a_y + 1 \right)
    \end{equation*}
    with energy eigenvalues $E_n = \hbar\omega\left( n+1 \right)$ where $n=n_x+n_y$. To uniquely label eigenstates, we could use $\left| n,n_x \right\rangle$ or use $L_z$,
    \begin{equation*}
        \left[ L_z,H \right] = 0,
    \end{equation*}
    as done previously. That is, we can find basis with fixed energy and angular momentum for each eigenstate.

    \np The states $\left| n_x \right\rangle\left| n_y \right\rangle$ are not eigenstates of $L_z$:
    \begin{equation*}
        L_z\left| n_x \right\rangle\left| n_y \right\rangle = \left( xp_y-yp_x \right)\left| n_x \right\rangle\left| n_y \right\rangle,
    \end{equation*}
    where
    \begin{equation*}
        xp_y-yp_x = i\hbar\left( a_xa_y^{\dagger}-a_x^{\dagger}a_y \right).
    \end{equation*}
    Observe
    \begin{equation*}
        a_x\left| n_x \right\rangle\left| n_y \right\rangle = \sqrt{n_x}\left| n_x-1 \right\rangle\left| n_y \right\rangle, a_x^{\dagger} = \sqrt{n_x+1}\left| n_x+1 \right\rangle\left| n_y \right\rangle
    \end{equation*}
    and
    \begin{equation*}
        a_y\left| n_x \right\rangle\left| n_y \right\rangle = \sqrt{n_y}\left| n_x \right\rangle\left| n_y-1 \right\rangle, a_y^{\dagger} = \sqrt{n_y+1}\left| n_x \right\rangle\left| n_y+1 \right\rangle.
    \end{equation*}
    Therefore,
    \begin{equation*}
        L_z\left| n_x \right\rangle\left| n_y \right\rangle = i\hbar \left( \sqrt{n_x}\sqrt{n_y+1}\left| n_x-1 \right\rangle\left| n_y+1 \right\rangle - \sqrt{n_x+1}\sqrt{n_y}\left| n_x+1 \right\rangle\left| n_y-1 \right\rangle \right) .
    \end{equation*}
    Observe that $\left| n_x-1 \right\rangle\left| n_y+1 \right\rangle,\left| n_x+1 \right\rangle\left| n_y-1 \right\rangle$ are orthonormal to $\left| n_x \right\rangle\left| n_y \right\rangle$, so $\left| n_x \right\rangle\left| n_y \right\rangle$ is not an eigenstate of $L_z$.

    \np To make things work out, we'd like to use different ladder operators $a_1,a_2$ such that
    \begin{equation*}
        H = \hbar\omega\left( a_1^{\dagger}a_1+a_2^{\dagger}a_2+1 \right)
    \end{equation*}
    and
    \begin{equation*}
        L_z = \alpha a^{\dagger}_1a_1 + \beta a_2^{\dagger}a_2.
    \end{equation*}
    If we can find such ladder operators, then $\hbar\omega\left( n_1+n_2+1 \right)$ and $\alpha n_1+\beta n_2$ are the eigenvalue of $H,L_z$, respectively, corresponding to $\left| n_1 \right\rangle\left| n_2 \right\rangle$. Of course, $a_1,a_2$ should satisfy \textit{obvious} commutation relations for ladder operators,
    \begin{equation*}
        \left[ a_1,a_1^{\dagger} \right] = 1 \left[ a_2,a_2^{\dagger} \right]
    \end{equation*}
    and
    \begin{equation*}
        \left[ a_1,a_2 \right] = \left[ a_1,a_2^{\dagger} \right] = \left[ a_1^{\dagger},a_2 \right] = \left[ a_1^{\dagger},a_2^{\dagger} \right] = 0.
    \end{equation*}

    \np Define
    \begin{equation*}
        a_1 = \frac{1}{\sqrt{2}} \left( a_x-ia_y \right), a_2 = \frac{1}{\sqrt{2}} \left( a_x+ia_y \right).
    \end{equation*}
    Then
    \begin{equation*}
        \left[ a_1,a_1^{\dagger} \right] = \left[ \frac{1}{\sqrt{2}}\left( a_x-ia_y \right),\frac{1}{\sqrt{2}}\left( a_x^{\dagger}+ia_y^{\dagger} \right) \right] = \frac{1}{2} \left( \left[ a_x,a_x^{\dagger} \right] - i\left[ a_y,a_x^{\dagger} \right]+i\left[ a_x,a_y^{\dagger} \right]+\left[ a_y,a_y^{\dagger} \right] \right) = 1.
    \end{equation*}
    In a similar manner, $a_1,a_2$ satisfy other commutation relations.

    Moreover, observe that
    \begin{equation*}
        \begin{aligned}
            a_1^{\dagger}a_1 & = \frac{1}{2}\left( a_x^{\dagger}+ia_y^{\dagger} \right)\left( a_x-ia_y \right) = \frac{1}{2} \left( a_x^{\dagger}a_x+a_y^{\dagger}a_y+ia_y^{\dagger}a_x-ia_x^{\dagger}a_y \right), \\
            a_2^{\dagger}a_2 & = \frac{1}{2}\left( a_x^{\dagger}-ia_y^{\dagger} \right)\left( a_x+ia_y \right) = \frac{1}{2} \left( a_x^{\dagger}a_x+a_y^{\dagger}a_y-ia_y^{\dagger}a_x+ia_x^{\dagger}a_y \right).
        \end{aligned} 
    \end{equation*}
    This means
    \begin{equation*}
        \begin{aligned}
            a_1^{\dagger}a_1 + a_2^{\dagger}a_2 & = a_x^{\dagger}a_x + a_y^{\dagger}a_y, \\
            a_1^{\dagger}a_1 - a_2^{\dagger}a_2 & = ia_x^{\dagger}a_x - ia_y^{\dagger}a_y.
        \end{aligned} 
    \end{equation*}
    It follows
    \begin{equation*}
        \begin{aligned}
            L_z & = \hbar\left( a_1^{\dagger}a_1 - a_2^{\dagger}a_2 \right), \\
            H & = \hbar\omega\left( a_1^{\dagger}a_1+a_2^{\dagger}a_2+1 \right) .
        \end{aligned} 
    \end{equation*}
    This means the eigenvalues of $L_z,H$ corresponding to $\left| n_1 \right\rangle\left| n_2 \right\rangle$ are $\hbar\left( n_1-n_2 \right), \hbar\omega\left( n_1+n_2+1 \right)$.

    \np Consider spherically symmetric Hamiltonian
    \begin{equation*}
        H = \frac{p^{2}}{2m} + V\left( r \right) = -\frac{\hbar^{2}}{2m}\nabla^{2}+V\left( r \right).
    \end{equation*}
    Also recall
    \begin{equation}
        \nabla^{2} = \frac{1}{r^{2}} \partial_r \left( r^{2}\partial_r \right) + \frac{1}{r^{2}\sin\left( \theta \right)} \partial_{\theta}\left( \sin\left( \theta \right)\partial_{\theta} \right) + \frac{1}{r^{2}\sin\left( \theta \right)^{2}} \partial_{\phi}^{2}
    \end{equation} 
    with respect to the spherical coordinates. Let's compare [1.13] with
    \begin{equation*}
        L^{2} = -\hbar^{2}\left( \partial_{\theta}^{2}+\cot\partial_{\theta}+\frac{1}{\sin\left( \theta \right)^{2}}\partial_{\phi}^{2} \right).
    \end{equation*}
    Observe that
    \begin{equation*}
        \frac{1}{\sin\left( \theta \right)}\partial_{\theta}\left(\sin\left( \theta \right)\partial_{\theta}\right) = \cdots = \cot\left( \theta \right) \partial_{\theta} + \partial_{\theta}^{2}.
    \end{equation*}
    Consequently,
    \begin{equation*}
        \nabla^{2} = \frac{1}{r^{2}}\partial_r\left( r^{2}\partial_r \right) - \frac{1}{\hbar^{2}r^{2}}L^{2}.
    \end{equation*}
    Hence
    \begin{equation*}
        H = \underbrace{-\frac{\hbar^{2}}{2mr^{2}}\partial_r\left( r^{2}\partial_r \right)}_{\text{radial kinetic energy}} + \underbrace{\frac{1}{2mr^{2}}L^{2}}_{\clap{\text{\scriptsize rotational energy}}} + V\left( r \right).
    \end{equation*}
    To find eigenfunctions $\psi$ of $H$, we use an ansatz
    \begin{equation*}
        \psi\left( r,\theta,\phi \right) = R\left( r \right)Y_l^m\left( \theta,\phi \right).
    \end{equation*}
    Then
    \begin{equation*}
        L^{2}Y_l^m\left( \theta,\phi \right) = \hbar^{2}l\left( l+1 \right)Y_l^m\left( \theta,\phi \right).
    \end{equation*}
    Observe that the equtaion
    \begin{equation*}
        H\psi = E_n\psi
    \end{equation*}
    becomes
    \begin{equation*}
        -\frac{\hbar^{2}}{2mr^{2}}\partial_r\left( r^{2}\partial_r \right) RY_l^m + \underbrace{\frac{1}{2mr^{2}}L^{2}RY_l^m}_{=\frac{\hbar^{2}l\left( l+1 \right)}{2mr^{2}}RY_l^m} + VRY_l^m = E_nRY_l^m.
    \end{equation*}
    But we are dealing with a spherically symmetrical system, so that we may cancel out $Y_l^m$ and obtain
    \begin{equation}
        -\frac{\hbar^{2}}{2mr^{2}}\partial_r\left( r^{2}\partial_r \right) R + \frac{\hbar^{2}l\left( l+1 \right)}{2mr^{2}}R+ VR = E_nR.
    \end{equation}
    Hence the radial wavefunction $R$ depends on $E_n$ and $l$, so we are going to index $R$ with $n,l$.

    Using change of variables $u = rR$,
    \begin{equation*}
        \frac{dR}{dr} = \frac{d}{dr} \left(\frac{u}{r}\right) = \frac{1}{r} \frac{du}{dr} - \frac{u}{r^{2}} = \frac{1}{r^{2}}\left( r \frac{du}{dr}-u \right).
    \end{equation*}
    Hence
    \begin{equation*}
        \frac{d}{dr}\left( r^{2} \frac{dR}{dr} \right) = \frac{d}{dr}\left( r \frac{du}{dr}-\frac{u}{r^{2}}  \right) = \frac{du}{dr} + r \frac{d^{2}u}{dr^{2}} - \frac{du}{dr} = r \frac{d^{2}u}{dr^{2}}.
    \end{equation*}
    Using this, [1.14] can be written as
    \begin{equation}
        -\frac{\hbar^{2}}{2m} \frac{d^{2}u}{dr^{2}} + \left( V + \frac{\hbar^{2}}{2mr^{2}}l\left( l+1 \right) \right)u = E_nu.
    \end{equation}
    [1.15] looks like energy eigenvalue equation for 1-dimensional Hamiltonian with \textit{effective} potential energy
    \begin{equation*}
        V + \underbrace{\frac{\hbar^{2}}{2mr^{2}l\left( l+1 \right)}}_{\clap{\text{\scriptsize centrifugal term}}}.
    \end{equation*}
    We call [1.15] the \textit{radial equation for central potential}.

    \begin{summary}{Eigenfunctions for Central Potential}
        The eigenfunctions for central potential has the form
        \begin{equation*}
            \psi_{nlm}\left( r,\theta,\phi \right) = R_{nl}\left( r \right) Y_l^m \left( \theta,\phi \right),
        \end{equation*}
        where $n$ is the energy state, $l$ is the quantum number in $L^{2}$ eigenvalue $\hbar l\left( l+1 \right)$ and $m$ is the quantum number in $L_z$ eigenvalue $\hbar m$. Assuming a spherically symmetrical system, we have a differential equation called \textit{radial eqaution}
        \begin{equation*}
            -\frac{\hbar^{2}}{2m} \frac{d^{2}u}{dr^{2}} + \left( V + \frac{\hbar^{2}}{2mr^{2}}l\left( l+1 \right) \right)u = E_nu.
        \end{equation*}
    \end{summary}

    \begin{example}{Hydrogen Atom}
        Hydrogen atom has potential
        \begin{equation*}
            V\left( r \right) = -\frac{e^{2}}{4\pi\epsilon_0} \frac{1}{r},
        \end{equation*}
        where $r$ is the distance between proton and electron.

        The radial equation becomes
        \begin{equation*}
            -\frac{\hbar^{2}}{2m} \frac{d^{2}u}{dr^{2}} + \left( -\frac{e^{2}}{4\pi\epsilon_0} \frac{1}{r} + \frac{\hbar^{2}}{2mr^{2}}l\left( l+1 \right) \right) u = E_nu.
        \end{equation*}
        Rearranging,
        \begin{equation*}
            \frac{d^{2}u}{dr^{2}} + \left( \frac{2me^{2}}{4\pi\epsilon_0\hbar^{2}} \frac{1}{r} - \frac{1}{r^{2}} l\left( l+1 \right) \right) u = \frac{-2mE_n}{\hbar^{2}} u.
        \end{equation*}
        Let
        \begin{equation*}
            k = \sqrt{\frac{-2mE_n}{\hbar^{2}}}
        \end{equation*}
        so that
        \begin{equation*}
            E_n = \frac{-\hbar^{2}k^{2}}{2m}.
        \end{equation*}
        Observe that $\hbar k$ is momentum and
        \begin{equation*}
            k = \frac{2\pi}{\lambda},
        \end{equation*}
        where $\lambda$ is the \textit{de Broglie wavelength}. Also, recall the \textit{Bohr radius}
        \begin{equation*}
            a_0 = \frac{4\pi\epsilon_0\hbar^{2}}{me^{2}}.
        \end{equation*}
        Using the introduced constants and variables, the radial eqaution can be written as
        \begin{equation}
            \frac{d^{2}u}{dr^{2}} + \left( \frac{2}{a_0r}-\frac{l\left( l+1 \right)}{r^{2}} \right) u = k^{2}u.
        \end{equation}

        We are going to attact [1.16] by
        \begin{enumerate}
            \item finding dimensionless variables;
            \item looking at asymtotic behavior; and
            \item hypothesizing power series and finding relation between coefficients.
        \end{enumerate}
        First, we introduce the \textit{dimensionless} version of $r$,
        \begin{equation*}
            \rho = kr,
        \end{equation*}
        to turn [1.16] into
        \begin{equation}
            \frac{d^{2}u}{d\rho^{2}} = u\left( 1-\frac{\rho_0}{\rho}+\frac{l\left( l+1 \right)}{\rho^{2}} \right).
        \end{equation}
        When we express $u$ as a power series in $\rho$, it does not play well with limits $\rho\to 0$ or $\rho\to\infty$. We hence look at the asymtotic behaviors of the solutions.

        In limit $\rho\to\infty$, [1.17] becomes
        \begin{equation*}
            \frac{d^{2}u}{d\rho^{2}} = u \implies u = Ae^{-\rho} + Be^{\rho}.
        \end{equation*}
        But $e^{\rho}$ blows up to $\infty$ as $\rho\to\infty$, so we conclude $u\to Ae^{-\rho}$ as $\rho\to\infty$.

        In the limit $\rho\to 0$,
        \begin{equation*}
            \frac{d^{2}u}{d\rho^{2}} = \frac{l\left( l+1 \right)}{\rho^{2}}u \implies u = C\rho^{l+1} + D\rho^{-l}.
        \end{equation*}
        But $\rho^{-l}\to\infty$ as $\rho\to 0$, it follows $u\to C\rho^{l+1}$ as $\rho\to 0$.

        Hence consider the ansatz
        \begin{equation*}
            u\left( \rho \right) = \rho^{l+1}e^{-\rho}v\left( \rho \right)
        \end{equation*}
        for some power series $v\left( \rho \right)$ in $\rho$.

        Observe that
        \begin{equation*}
            \frac{du}{d\rho} = \rho^le^{-\rho}\left( v\left( \rho \right)\left( l+1-\rho \right)+\rho \frac{dv}{d\rho} \right)
        \end{equation*}
        and
        \begin{equation}
            \frac{d^{2}u}{d\rho^{2}} = \rho^le^{-\rho}\left( v\left( \rho \right)\left( \frac{l\left( l+1 \right)}{\rho}-2l-2+\rho \right)+\frac{dv}{d\rho}\left( 2\left( l+1-\rho \right) \right)+\frac{d^{2}v}{d\rho^{2}}\rho \right).
        \end{equation}
        Plugging [1.18] into [1.17] gives
        \begin{equation}
            \rho \frac{d^{2}v}{d\rho^{2}} + 2\left( l+1-\rho \right) \frac{dv}{d\rho} + \left( \rho_0-2l-2 \right) = 0.
        \end{equation}
        Write $v$ explicitly:
        \begin{equation*}
            v\left( \rho \right) = \sum^{\infty}_{j=0}c_j\rho^j
        \end{equation*}
        to find out the recurrence relation between the coefficients. Then
        \begin{equation*}
            \frac{dv}{d\rho} = \sum^{\infty}_{j=1} c_jj\rho^{j-1} = \sum^{\infty}_{j=0} \left( j+1 \right)c_{j+1}\rho^j
        \end{equation*} 
        and
        \begin{equation*}
            \frac{d^{2}v}{d\rho^{2}} = \sum^{\infty}_{j=0} j\left( j+1 \right)c_{j+1}\rho^{j-1.}
        \end{equation*}
        Using this, [1.19] becomes
        \begin{equation}
            \sum^{\infty}_{j=0} \left( j\left( j+1 \right)c_{j+1}+2\left( l+1 \right)\left( j+1 \right)c_{j+1}-2jc_j+\left( \rho_0-2l-2 \right)c_j \right)\rho^j = 0.
        \end{equation}
        This means
        \begin{equation}
            \left( j\left( j+1 \right)+2\left( l+1 \right)\left( j+1 \right) \right)c_{j+1} + \left( \rho_0-2l-2-2j \right)c_j = 0, \hspace{2cm}\forall j\geq 0
        \end{equation}
        or
        \begin{equation*}
            c_{j+1} = \frac{2j+2l+2-\rho_0}{\left( j+1 \right)\left( j+2l+2 \right)} c_j, \hspace{2cm}\forall j\geq 0
        \end{equation*}
        is the recurrence relation. 

        As $j\to\infty$,
        \begin{equation*}
            c_{j+1} \to \frac{2j}{\left( j+1 \right)j}c_j = \frac{2}{\left( j+1 \right)}c_j,
        \end{equation*}
        so that
        \begin{equation*}
            c_{j+1} \to \frac{2^{j+1}}{\left( j+1 \right)!}c_0.
        \end{equation*}
        This means
        \begin{equation*}
            v\left( \rho \right) \approx e^{2\rho} \implies u\left( \rho \right) = \rho^{l+1}e^{-\rho}Ce^{2\rho} = C\rho^{l+1}e^{\rho}.
        \end{equation*}
        But this $v\left( \rho \right)$ is not normalizable, so there is $j_{\text{max}}$ for which $c_{j_{\text{max}}+1} = 0$. Then
        \begin{equation*}
            0 = c_{j_{\text{max}}+1} = \frac{2j_{\text{max}}+2l+2-\rho_0}{\left( j_{\max}+1 \right)\left( j_{\max}+2l+2 \right)}c_{j_{\max}} \implies 2j_{\max}+2l+2-\rho_0 = 0.
        \end{equation*}
        It follows that
        \begin{equation*}
            \rho_0 = 2\left( j_{\max}+l+1 \right) \implies n = j_{\max}+l+1.
        \end{equation*}
        Recalling
        \begin{equation*}
            \rho_0 = \frac{me^{2}}{2\pi\epsilon_0\hbar^{2}k}, k = \frac{\sqrt{-2mE}}{\hbar},
        \end{equation*}
        we get
        \begin{equation*}
            E_n = \frac{-\hbar^{2}k^{2}}{2m} = -\frac{\hbar^{2}}{2m}\left( \frac{me^{2}}{2\pi\epsilon_0\hbar^{2}\rho_0} \right)^{2} = \frac{-\hbar^{2}}{2m}\left( \frac{me^{2}}{2\pi\epsilon_0\hbar^{2}} \right)\frac{1}{\left( 2n \right)^{2}} = -\frac{R_0}{n^{2}},
        \end{equation*}
        where
        \begin{equation*}
            R_0 = \frac{m}{2\hbar^{2}} \left( \frac{e^{2}}{4\pi\epsilon_0} \right)^{2} \approx 13.6\text{eV}
        \end{equation*}
        is the \textit{Rydberg energy}, the energy required to \textit{unbind} the electron from a ground state hydrogen atom.

        Here is a summary of the relations for the allowed quantum numbers $n,l,m_l$, which index the eigenvalues $E_n, \hbar^{2}l\left( l+1 \right), \hbar m_l$ of $H,L^{2},L_z$:
        \begin{equation*}
            n\text{ is a positive integer}, n=j_{\max}+l+1, n\geq l+1, -l\leq m_l\leq l.
        \end{equation*}
        Moreover,
        \begin{equation*}
            R_{nl}\left( r \right) = a_0 + a_1r + a_2r^2 + \cdots + a_{j_{\max}}r^{j_{\max}}.
        \end{equation*}
    \end{example}

    \rruleline

    \begin{example}{Degeneracy}
        For a hydrogen atom, when $n=3$, what are the number of degeneracies?
    \end{example}

    \begin{answer}
        Since $n=3$, the allowed values of $l$ are $2,1,0$. For $l=2$, there are $5$ allowed values for $m_l$; for $l=1$, there are $3$ allowed values for $m_l$; and for $l=0$ there is only one allowed value for $m_l$.

        Thus there are $9$ degeneracies for $n=3$.
    \end{answer}

    \subsection{Continuous Wavefunctions}
    
    Suppose that a $\left| \psi \right\rangle$ in an infinite-dimensional Hilbert space $\mH$ and a position operator $\hat{x}$ are given. Then for any position eigenvalue $x$ of $\hat{x}$, we define
    \begin{equation*}
        \psi\left( x \right) = \left\langle x | \psi \right\rangle.
    \end{equation*}
    In other words, the function $\psi\left( \cdot \right)$ is a representation of $\left| \psi \right\rangle$ in an infinite-dimensional space with a fixed ONB.

    The orthonormality condition and completeness relation are modified to
    \begin{equation*}
        \left\langle n | m \right\rangle = \delta_{n,m} \to \left\langle x | x' \right\rangle = \delta\left( x-x' \right) \eqno\text{\textit{orthonormality}}
    \end{equation*}
    and
    \begin{equation*}
        \sum^{}_{n}\left| n \right\rangle\left\lbrace n \right\rbrace = I \to \int^{\infty}_{-\infty}\left| x \right\rangle \left\langle x \right|dx = I, \eqno\text{\textit{completeness relation}}
    \end{equation*}
    where $\delta\left( \cdot \right)$ is the \textit{Direc delta distribution}.

    Using completeness relation, we can expand $\left| \psi \right\rangle$ in $\left\lbrace \left| x \right\rangle \right\rbrace^{}_{x}$ basis:
    \begin{equation*}
        \left| \psi \right\rangle = \int^{\infty}_{-\infty}\left| x \right\rangle \left\langle x | \psi \right\rangle dx = \int^{\infty}_{-\infty}\psi\left( x \right)\left| x \right\rangle dx 
    \end{equation*}
    analogous to the finite-dimensional case.

    \np Similar to how we use coefficients $c_n$'s of $\left| \psi \right\rangle$ in an ONB to find out useful quantities about $\left| \psi \right\rangle$, we can use $\psi\left( \cdot \right)$ to find calculate inner products, probabilities, expectation values, and so on. For instance,
    \begin{equation*}
        \left\langle \phi | \psi \right\rangle = \int^{\infty}_{-\infty} \phi\left( x \right)^{*}\psi\left( x \right)dx. \eqno\text{\textit{inner product}}
    \end{equation*}
    When we stick in an operator in between, 
    \begin{equation*}
        \left\langle \phi \right|A\left| \psi \right\rangle = \left\langle \phi \right| IAI \left| \psi \right\rangle = \int^{\infty}_{-\infty}\int^{\infty}_{-\infty} \left\langle \phi | x \right\rangle \left\langle x \right|A\left| y \right\rangle \left\langle y | \psi \right\rangle dxdy.
    \end{equation*}
    For simplicity, consider the case where $A = \hat{x}$, the position operator. Then
    \begin{equation*}
        \left\langle \phi \right|\hat{x}\left| \psi \right\rangle = \int^{\infty}_{-\infty}\int^{\infty}_{-\infty} \left\langle \phi | x \right\rangle \left\langle x \right|\hat{x}\left| y \right\rangle \left\langle y | \psi \right\rangle dxdy = \int^{\infty}_{-\infty}\int^{\infty}_{-\infty} \phi\left( x \right)^{*}y\delta\left( x-y \right)\psi\left( x \right)dxdy = \int^{\infty}_{-\infty} \phi\left( x \right)^{*}x\psi\left( x \right)dx.
    \end{equation*}
    More generally,
    \begin{equation*}
        \left\langle \phi \right|\hat{A}\left| \psi \right\rangle = \int^{\infty}_{-\infty}\phi\left( x \right)^{*}A\left( x \right)\psi\left( x \right)dx.
    \end{equation*}

    \np We interprete $\psi$ as the \textit{probability density} (or to be more precisely, $\left| \phi \right|^{2}$ is a pdf). That is,
    \begin{equation*}
        \int^{\infty}_{-\infty}\left| \psi\left( x \right) \right|^{2}dx = 1.
    \end{equation*}

    \subsection{AdditionofAngular Momentum}
    
    Suppose $\vec{J}_1, \vec{J}_2$ are angular momenta. Consider $\vec{J} = \vec{J}_1+\vec{J}_2$.\footnote{Since $\vec{J}_1,\vec{J}_2$ are angular momenta, they belong to \textit{different systems}, so more precisely $\vec{J} = \vec{J}_1\otimes I_2 + I_1\otimes \vec{J}_2$.} Then $\vec{J}$ obeys the same commutation relation
    \begin{equation*}
        \left[ J^i,J^j \right] = i\hbar J_k\epsilon_{ijk}, \hspace{2cm}\forall i,j,k.
    \end{equation*}
    What basis states should we use?

    An \textit{obvious} choice is uncoupled basis. That is, the common eigenbasis of $J_1^{2},J_2^{2},J_{1,z},J_{2,z}$, with
    \begin{equation*}
        \begin{aligned}
            J_1^{2}\left| j_1,j_2,m_1,m_2 \right\rangle & = \hbar^{2}j_1\left( j_1+1 \right)\left| j_1,j_2,m_1,m_2 \right\rangle, \\
            J_2^{2}\left| j_1,j_2,m_1,m_2 \right\rangle & = \hbar^{2}j_2\left( j_2+1 \right)\left| j_1,j_2,m_1,m_2 \right\rangle, \\
            J_{1,z}\left| j_1,j_2,m_1,m_2 \right\rangle & = \hbar m_1\left| j_1,j_2,m_1,m_2 \right\rangle, \\
            J_{2,z}\left| j_1,j_2,m_1,m_2 \right\rangle & = \hbar m_2\left| j_1,j_2,m_1,m_2 \right\rangle,
        \end{aligned} 
    \end{equation*}
    where
    \begin{equation*}
        \left| j_1,j_2,m_1,m_2 \right\rangle = \left| j_1,m_1 \right\rangle\left| j_2,m_2 \right\rangle.
    \end{equation*}

    Another choice is better if $J_1,J_2$ interact: common eigenbasis of $J_1^{2},J_2^{2},J^{2},J_z^{2}$.
    \begin{equation*}
        \begin{aligned}
            J_1^{2}\left| j_1,j_2,j,m \right\rangle & = \hbar^{2}j_1\left( j_1+1 \right)\left| j_1,j_2,j,m \right\rangle, \\
            J_2^{2}\left| j_1,j_2,j,m \right\rangle & = \hbar^{2}j_2\left( j_2+1 \right)\left| j_1,j_2,j,m \right\rangle, \\
            J^{2}\left| j_1,j_2,j,m \right\rangle & = \hbar^{2}j\left( j+1 \right)\left| j_1,j_2,j,m \right\rangle, \\
            J_z^{2}\left| j_1,j_2,j,m \right\rangle & = \hbar m\left| j_1,j_2,j,m \right\rangle. \\
        \end{aligned} 
    \end{equation*}
    
    Often we write
    \begin{equation*}
        \begin{aligned}
            \left| m_1,m_2 \right\rangle & = \left| j_1,j_2,m_1,m_2 \right\rangle \\
            \left| j,m \right\rangle & = \left| j_1,j_2,j,m \right\rangle \\
        \end{aligned} 
    \end{equation*}
    provided we know $j_1,j_2$.

    Then we know
    \begin{equation*}
        \left| j,m \right\rangle = \sum^{}_{m_1,m_2} \left\langle m_1,m_2 | j,m \right\rangle\left| m_1,m_2 \right\rangle.
    \end{equation*}
    The coefficients $\left\langle m_1,m_2 | j,m \right\rangle$ are caled the \textit{Clebsch-Gorden coefficient} of $\left| j,m \right\rangle$.

    The following selection rules tell us when the coefficients are nonzero.

    \begin{fact}{Selection Rule I}
        $m = m_1+m_2$ or $\left\langle m_1,m_2 | j,m \right\rangle = 0$.
    \end{fact}

    \clearpage

    \begin{example}{}
        Consider $j_1=2,j_2=1$. Then there are $\left( 2j_1+1 \right)\left( 2j_2+1 \right) = 15$ total basis states.

        The allowed values of $m$ are $-3,\ldots,3$ and we know $-j\leq m\leq j$. This means $j=3,2,1$ must be allowed, and we have
        \begin{equation*}
            \begin{aligned}
                j&=3 \implies \text{7 total states} \\
                j&=2 \implies \text{5 total states} \\
                j&=1 \implies \text{3 total states}\\
            \end{aligned} 
        \end{equation*}
    \end{example}
    
    \rruleline

    \np Hence here is an upgraded version of the selection rule

    \begin{fact}{Selection Rule II}
        $m = m_1+m_2$, $\left| j_1-j_2 \right|\leq j\leq j_1+j_2$ or $\left\langle m_1,m_2 | j,m \right\rangle = 0$.
    \end{fact}

    \np Consider writing
    \begin{equation*}
        \left| j,m \right\rangle = \sum^{}_{m_1,m_2} \left\langle m_1,m_2 | j,m \right\rangle\left| m_1,m_2 \right\rangle.
    \end{equation*}
    To find the coefficients, start with the state that has maximum projection along $z$:
    \begin{equation*}
        \left| m_1=j_1,m_2=j_2 \right\rangle = \left| j=j_1+j_2, m=j_1+j_2 \right\rangle,
    \end{equation*}
    which is unique in both bases.

    \np Start with
    \begin{equation*}
        \left| j=1,m=1 \right\rangle = \left| m_1=\frac{1}{2},m_2=\frac{1}{2} \right\rangle.
    \end{equation*}
    Using
    \begin{equation*}
        \begin{aligned}
            J_- & = J_{1,-} + J_{2,-} ,
            J_{\alpha,-}\left| j_{\alpha},m_{\alpha} \right\rangle & = \hbar\sqrt{j_{\alpha}\left( j_{\alpha}+1 \right)-m_{\alpha}\left( m_{\alpha}-1 \right)} \left| j_{\alpha},m_{\alpha}-1 \right\rangle, \\
        \end{aligned} 
    \end{equation*}
    we get
    \begin{equation*}
        \begin{aligned}
            J_{-} & = J_{1,-}\left| m_1=\frac{1}{2},m_2=\frac{1}{2} \right\rangle + J_{2,-}\left| m_1=\frac{1}{2},m_2=\frac{1}{2} \right\rangle  \\
                  & = \hbar\sqrt{\frac{1}{2}\left( \frac{1}{2}+1 \right)-\frac{1}{2}\left( \frac{1}{2}-1 \right)}\left( \left| m_1=-\frac{1}{2},m_2=\frac{1}{2} \right\rangle+\left| m_1=\frac{1}{2},m_2=-\frac{1}{2} \right\rangle \right).
        \end{aligned} 
    \end{equation*}
    It follows
    \begin{equation*}
        \left| j=1,m=0 \right\rangle = \frac{1}{\sqrt{2}} \left( \left| m_1=-\frac{1}{2},m_2=\frac{1}{2} \right\rangle+\left| m_1=\frac{1}{2},m_2=-\frac{1}{2} \right\rangle \right).
    \end{equation*}





























    
    
    
    
    
    
    
    
    

\end{document}
